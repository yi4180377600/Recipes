\documentclass{article}
\usepackage{geometry}
\usepackage[T1]{fontenc}
\usepackage[utf8]{inputenc}
\usepackage[english,pinyin]{babel}
\babelprovide[main,import,language=Default]{english}
\babelprovide[import,language=Chinese Simplified]{chinese-simplified}
\babelfont[chinese-simplified]{rm}{SimSun}
\babelfont[chinese-simplified]{sf}{STKaiti}
\usepackage{paracol}
\usepackage{hyperref}
\setlength{\columnseprule}{0.5pt}
\usepackage{indentfirst}
\setlength{\parindent}{1.5em}
\footnotelayout{m}
\newcommand{\chn}[1]{\foreignlanguage{chinese-simplified}{#1}}


\begin{document}
\title{My Recipes}
\author{Yi}
\date{\today}
\maketitle
\begin{abstract}
    \chn{\textsf{个人尝试过}并且效果不错的一些菜谱,以及其参考教程和英文翻译。}
    Some recipes \emph{I have personally tried} with good results, together with their reference tutorials and English translations.
\end{abstract}
\tableofcontents
\section{\foreignlanguage{chinese-simplified}{红烧排骨} Red-Braised Pork Ribs}
Reference: \url{https://m.xiachufang.com/recipe/104038857/}.

Ingredients: 
\begin{itemize}
    \item \chn{排骨 (500g,2至3人份),生抽,白糖,料酒,生姜,香叶,桂皮}
    \item Pork ribs (500g for 2-3), (light) soy sauce, white sugar, cooking wine, ginger, bay leaves, cinnamon stick \footnote{Bay leaves and cinnamon sticks can be substituted with other spices.}
\end{itemize}

\begin{paracol}{2}
    \begin{enumerate}
        \item \foreignlanguage{chinese-simplified}{将肋排切成小块,沿着骨头方向即可。}
        \item \chn{把排骨冷水下锅焯水,煮沸之后冲凉水将排骨洗净。}
        \item \chn{准备几片姜片,香叶一片,桂皮一块,泡在水中作为香料水备用。}
        \item \chn{炒糖色:在锅中加入油和2大勺白糖,中小火慢熬大约2分钟,同时沿一个方向搅拌,至白糖变成黄色泡沫状即可。}
        \item \chn{把排骨加入糖色中翻炒,使其裹上糖汁,之后加入香料水(包含香料)。}
        \item \chn{加入适量生抽和料酒,盖上锅盖中小火炖大约30分钟。}
        \item \chn{当汤汁基本收成酱汁时,翻炒一下排骨让其被酱汁包裹。}
    \end{enumerate}
    
    \switchcolumn
    \begin{enumerate}
        \item Cut the pork ribs into small pieces, along the direction of the bones.
        \item Place the pork ribs in a pot of cold water and bring to a boil. 
        Once boiling, rinse the ribs under cold water to clean them thoroughly.
        \item Prepare a few slices of ginger, one bay leaf, and a piece of cinnamon stick. 
        Soak them in water to make spiced water for later use.
        \item Caramelizing sugar\footnote{This step is to give the dish a rich, reddish glaze. If you don't aim for this effect, you can skip this step and add dark soy sauce later as a substitute.}: 
        Add oil and 2 tablespoons of white sugar to a pan. 
        Cook over medium-low heat for about 2 minutes, stirring in one direction until the sugar turns into a yellow, foamy texture.
        \item Add the pork ribs to the caramelized sugar and stir-fry until they are coated with the sugar syrup. 
        Then add the spiced water (including the spices).
        \item Add an appropriate amount\footnote{For reference: 500g of pork ribs can be seasoned with 3 teaspoons of light soy sauce and 2 teaspoons of cooking wine. If you're concerned about the dish being too salty, start with a smaller amount of soy sauce, and add more during the simmering process if the flavor is too bland.} 
        of light soy sauce and cooking wine. 
        Cover the pot and simmer over medium-low heat for about 30 minutes.
        \item When the sauce has reduced to a thick consistency, stir-fry the pork ribs to coat them with the sauce.
    \end{enumerate}
\end{paracol}
\section{\chn{虾仁滑蛋} Shrimp with Scrambled Eggs}
Reference: \url{https://www.youtube.com/watch?v=UB25StsHNXU}

Ingredients: 
\begin{itemize}
    \item \chn{鲜虾(500g, 2人份)或冷冻虾仁,鸡蛋(4-5个),白胡椒粉,盐,料酒,淀粉,葱}
    \item Fresh or frozen shrimp (500g for 2-3), eggs (4-5), white pepper, salt, cooking wine, starch, green onions
\end{itemize}
\begin{paracol}{2}
    \begin{itemize}
        \item \chn{鲜虾去壳去虾线。虾仁加少量盐和白胡椒给底味,抓匀。加入少量料酒去腥,以及少量淀粉锁水,抓匀。}
        \item \chn{鸡蛋打入碗中,加少量盐和白胡椒之后打散。如果想让鸡蛋更滑嫩,可以加入少量牛奶或水淀粉。}
        \item \chn{热锅三四成油温,下入虾仁。使虾仁间相互分开防止粘连。转动锅让油流动,让虾肉均匀受热,直到变色定型再推散。}
        \item \chn{将虾仁捞出放入蛋液里,锅内留少量底油。}
        \item \chn{把蛋液和虾仁同时下入锅中,用铲子慢慢地推动蛋液,尽量不要使其变为焦黄色,让蛋液尽可能包裹虾仁。}
        \item \chn{当大部分蛋液凝固,还有少量液体时,关火用余温再加热几秒。装盘并用葱花点缀。}
    \end{itemize}
    \switchcolumn
    \begin{itemize}
        \item Peel and devein\footnote{Removing the vein is a matter of personal preference and taste, not hygiene. It's not harmful to eat.} the shrimp. 
        Season the shrimp with a small amount of salt and white pepper, then mix well. 
        Add a little cooking wine to remove the fishy smell and some starch to retain the moisture of the shrimp, then mix evenly. 
        \item Crack the eggs into a bowl, add some salt and white pepper, then whisk well.
        To make the eggs smoother and more tender, you can add a little milk or cornstarch slurry.
        \item Heat the oil to 90-120 $^{\circ}$C and add the shrimp. 
        Separate the shrimp to prevent them from sticking together. 
        Swirl the pan to move the oil around, ensuring even cooking.
        Once the shrimp turn pink and firm up, gently stir them.
        \item Remove the shrimp and place them into the egg mixture. Leave some oil in the pan.
        \item Pour the egg mixture together with shrimp into the pan. 
        Gently stir and push the eggs with a spatula, avoiding browning. 
        Try to coat the shrimp with the egg as much as possible.
        \item When most of the egg mixture has set but some liquid remains,
        turn off the heat and let the residual heat cook it for a few more seconds.
        Plate the dish and garnish with chopped green onions.
    \end{itemize}
\end{paracol}
\end{document}