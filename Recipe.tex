\documentclass{article}
\usepackage{geometry}
\usepackage[T1]{fontenc}
\usepackage[utf8]{inputenc}
\usepackage[english,pinyin]{babel}
\babelprovide[main,import,language=Default]{english}
\babelprovide[import,language=Chinese Simplified]{chinese-simplified}
\babelfont[chinese-simplified]{rm}{SimSun}
\babelfont[chinese-simplified]{sf}{STKaiti}
\usepackage{paracol}
\usepackage{hyperref}
\setlength{\columnseprule}{0.5pt}
\usepackage{indentfirst}
\setlength{\parindent}{1.5em}
\footnotelayout{m}
\newcommand{\chn}[1]{\foreignlanguage{chinese-simplified}{#1}}


\begin{document}
\title{My Recipes}
\author{Yi}
\date{\today}
\maketitle
\begin{abstract}
    \chn{\textsf{个人尝试过}并且效果不错的一些菜谱,以及其参考教程和英文翻译。}
    Some recipes \emph{I have personally tried} with good results, together with their reference tutorials and English translations.
\end{abstract}
\tableofcontents
\section{\foreignlanguage{chinese-simplified}{红烧排骨} Red-Braised Pork Ribs}
Reference: \url{https://m.xiachufang.com/recipe/104038857/}.

Ingredients: 
\begin{itemize}
    \item \chn{排骨,生抽,白糖,料酒,生姜,香叶,桂皮}
    \item Pork ribs, (light) soy sauce, white sugar, cooking wine, ginger, bay leaves, cinnamon stick \footnote{Bay leaves and cinnamon sticks can be substituted with other spices.}
\end{itemize}

\begin{paracol}{2}
    \begin{enumerate}
        \item \foreignlanguage{chinese-simplified}{将肋排切成小块,沿着骨头方向即可。}
        \item \chn{把排骨冷水下锅焯水,煮沸之后冲凉水将排骨洗净。}
        \item \chn{准备几片姜片,香叶一片,桂皮一块,泡在水中作为香料水备用。}
        \item \chn{炒糖色:在锅中加入油和2大勺白糖,中小火慢熬大约2分钟,同时沿一个方向搅拌,至白糖变成黄色泡沫状即可。}
        \item \chn{把排骨加入糖色中翻炒,使其裹上糖汁,之后加入香料水(包含香料)。}
        \item \chn{加入适量生抽和料酒,盖上锅盖中小火炖大约30分钟。}
        \item \chn{当汤汁基本收成酱汁时,翻炒一下排骨让其被酱汁包裹。}
    \end{enumerate}
    
    \switchcolumn
    \begin{enumerate}
        \item Cut the pork ribs into small pieces, along the direction of the bones.
        \item Place the pork ribs in a pot of cold water and bring to a boil. 
        Once boiling, rinse the ribs under cold water to clean them thoroughly.
        \item Prepare a few slices of ginger, one bay leaf, and a piece of cinnamon stick. 
        Soak them in water to make spiced water for later use.
        \item Caramelizing sugar \footnote{This step is to give the dish a rich, reddish glaze. If you don't aim for this effect, you can skip this step and add dark soy sauce later as a substitute.}: 
        Add oil and 2 tablespoons of white sugar to a pan. 
        Cook over medium-low heat for about 2 minutes, stirring in one direction until the sugar turns into a yellow, foamy texture.
        \item Add the pork ribs to the caramelized sugar and stir-fry until they are coated with the sugar syrup. 
        Then add the spiced water (including the spices).
        \item Add an appropriate amount
        \footnote{For reference: 500g of pork ribs can be seasoned with 3 teaspoons of light soy sauce and 2 teaspoons of cooking wine. 
        If you're concerned about the dish being too salty, start with a smaller amount of soy sauce, and add more during the simmering process if the flavor is too bland.} 
        of light soy sauce and cooking wine. 
        Cover the pot and simmer over medium-low heat for about 30 minutes.
        \item When the sauce has reduced to a thick consistency, stir-fry the pork ribs to coat them with the sauce.
    \end{enumerate}
\end{paracol}
\section{\chn{虾仁滑蛋} Shrimp with Scrambled Eggs}

\end{document}